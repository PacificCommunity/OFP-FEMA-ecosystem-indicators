% Options for packages loaded elsewhere
\PassOptionsToPackage{unicode}{hyperref}
\PassOptionsToPackage{hyphens}{url}
%
\documentclass[
]{article}
\usepackage{amsmath,amssymb}
\usepackage{iftex}
\ifPDFTeX
  \usepackage[T1]{fontenc}
  \usepackage[utf8]{inputenc}
  \usepackage{textcomp} % provide euro and other symbols
\else % if luatex or xetex
  \usepackage{unicode-math} % this also loads fontspec
  \defaultfontfeatures{Scale=MatchLowercase}
  \defaultfontfeatures[\rmfamily]{Ligatures=TeX,Scale=1}
\fi
\usepackage{lmodern}
\ifPDFTeX\else
  % xetex/luatex font selection
\fi
% Use upquote if available, for straight quotes in verbatim environments
\IfFileExists{upquote.sty}{\usepackage{upquote}}{}
\IfFileExists{microtype.sty}{% use microtype if available
  \usepackage[]{microtype}
  \UseMicrotypeSet[protrusion]{basicmath} % disable protrusion for tt fonts
}{}
\makeatletter
\@ifundefined{KOMAClassName}{% if non-KOMA class
  \IfFileExists{parskip.sty}{%
    \usepackage{parskip}
  }{% else
    \setlength{\parindent}{0pt}
    \setlength{\parskip}{6pt plus 2pt minus 1pt}}
}{% if KOMA class
  \KOMAoptions{parskip=half}}
\makeatother
\usepackage{xcolor}
\usepackage[margin=1in]{geometry}
\usepackage{graphicx}
\makeatletter
\def\maxwidth{\ifdim\Gin@nat@width>\linewidth\linewidth\else\Gin@nat@width\fi}
\def\maxheight{\ifdim\Gin@nat@height>\textheight\textheight\else\Gin@nat@height\fi}
\makeatother
% Scale images if necessary, so that they will not overflow the page
% margins by default, and it is still possible to overwrite the defaults
% using explicit options in \includegraphics[width, height, ...]{}
\setkeys{Gin}{width=\maxwidth,height=\maxheight,keepaspectratio}
% Set default figure placement to htbp
\makeatletter
\def\fps@figure{htbp}
\makeatother
\setlength{\emergencystretch}{3em} % prevent overfull lines
\providecommand{\tightlist}{%
  \setlength{\itemsep}{0pt}\setlength{\parskip}{0pt}}
\setcounter{secnumdepth}{-\maxdimen} % remove section numbering
\ifLuaTeX
  \usepackage{selnolig}  % disable illegal ligatures
\fi
\IfFileExists{bookmark.sty}{\usepackage{bookmark}}{\usepackage{hyperref}}
\IfFileExists{xurl.sty}{\usepackage{xurl}}{} % add URL line breaks if available
\urlstyle{same}
\hypersetup{
  pdftitle={Potential fisheries related climate indicators for WCPFC `State of the Ocean' report},
  pdfauthor={nick hill},
  hidelinks,
  pdfcreator={LaTeX via pandoc}}

\title{Potential fisheries related climate indicators for WCPFC `State
of the Ocean' report}
\author{nick hill}
\date{2025-01-22}

\begin{document}
\maketitle

{
\setcounter{tocdepth}{2}
\tableofcontents
}
\clearpage

\hypertarget{introduction}{%
\section{1. Introduction}\label{introduction}}

The intent of this script is to provide some rough text and plots of
potential fisheries-related climate indicators to the group to stimulate
discussion. Its unlikely all of these will make the final report and are
here to provide options, discussion and an idea of the data available.
Please excuse any errors/imperfect plots etc. I'm hoping I've provided
enough text to explain what's been done but please ask if its not clear
(which is almost definitely the case).

We look at three main types of indicators which reflect both relevance
to climate change fisheries, and also the data we have available:

\begin{itemize}
\item
  Catch
\item
  Effort
\item
  Biology (length and weight-based)
\end{itemize}

One thing I have not yet explored is CPUE-related indicators. It'd be
good to get some feedback from others if we think this is worthwhile. It
is quite simple/achievable to explore some nominal-CPUE based
indicators, but I am unsure if this is relevant given the work done in
developing standardised CPUE indices. is there anything else we could
leverage from the stock assessments like recruitment indices?

\hypertarget{catch-related-indicators}{%
\section{2. Catch-related indicators}\label{catch-related-indicators}}

Here, we have extracted aggregated catch data for purse seine and
longline fisheries in the WCPFC from 1990-2023 for target species:

\begin{itemize}
\item
  ALB - albacore (north and south),
\item
  BET - Bigeye tuna,
\item
  SKJ - skipjack tuna (purse seine only),
\item
  YFT - yellowfin tuna.
\end{itemize}

We investigate a series of indicators or approaches to investigate
climate-related trends in catch and how we might monitor the fishery
using this data. This includes:

\begin{itemize}
\item
  Centre of gravity of PS catch,
\item
  Total area of PS catch,
\item
  Proportion of PS catch \textgreater{} 180 degrees longitude,
\item
  Proportion of LL catch beyond 20 degrees latitude.
\end{itemize}

\clearpage

\hypertarget{centre-of-gravity-of-ps-catch}{%
\subsection{2.1 Centre of gravity of PS
catch}\label{centre-of-gravity-of-ps-catch}}

Centre of gravity refers to the `centre' of fishing catch in terms of
both longitude and latitude. With the effects of climate change,
SEAPODYM has predicted that tuna biomass will shift eastwards with the
expansion of the warm pool. By monitoring the centre of gravity of
catch, we can track if this shift in biomass is occurring (assuming that
the fishery will follow the biomass). Catch and effort regularly shift
with ENSO events, so the hypothesis would be the PS fleet could shift in
relation to any climate-change driven changes.

Here, we extract the centre of gravity of catch which basically provides
a ``mean'' estimate of catch location (longitude and latitude) and an
intertia value which measures disperson (basically spread). We have
extracted these at annual timescales, but could be downgraded to
seasonal or monthly to explore within-year shifts also. Note: PS catch
data is filtered to WCPFC regions 6-8 (lat: -20:10 degrees, lon 140:210
degrees) to capture main PS fleet and filter out SE Asian domestic
effort (to be discussed).

First, we show a plot of catch density as a way of a summary of catch
spatially. We have done this for each species and both gears. As an
example, we have shown this for each species with historical data
(1990:2000) and the most recent year (2023) for each species as well as
effort. Note the distribution of longline catch in the density maps -
should we keep in SE Asian catch? This is dropped in some other
analyses.

Now we can explore COG for species. Here, we can see evidence of an
eastward shift in PS catch for bigeye tuna in particular, while for SKJ
and YFT it is less clear. It is also not clear if this is driven by
climate shifts, or a shift in effort, particularly sets associated with
FADs to the east that may be explaining this (to be discussed).

\begin{figure}
\centering
\includegraphics{./2024_analyses/results/combined/combined_PScatch_effort_density_map.png}
\caption{Density of purse seine catch and effort for historical period
(1990:2000) and 2023.}
\end{figure}

\begin{figure}
\centering
\includegraphics{./2024_analyses/results/catch/catch_density_LL_allsp.png}
\caption{Density of longline catch for historical period (1990:2000) and
2023.}
\end{figure}

\begin{figure}
\centering
\includegraphics{./2024_analyses/results/catch/catch_COGPS_map.png}
\caption{Centre of gravity of purse seine catch by species from
1990-2023.}
\end{figure}

\clearpage

\hypertarget{total-area-of-ps-and-longline-catch}{%
\subsection{Total area of PS and longline
catch}\label{total-area-of-ps-and-longline-catch}}

This indicator looks at the total area (km2) where species were
harvested. Climate change is likely to drive distribution shifts in
species as their environment changes with many species shifting
polewards as temperatures warm for example. Other species ranges can
change in size as their range edges can shift at different rates,
meaning that suitable habitat is increasing/decreasing at one edge
faster than the other for instance, or their distribution can become
more fragmented.

It is unknown how tuna species distribution is likely to change with
climate change in terms of total size, but it is predicted to shift
eastwards with the extension of the warm pool. Monitoring total catch
area could provide a monitor/early warning system that changes in the
distribution and abundance of tuna are occurring due to climate change.
Here, we can see the total catch area for each species by gear type. We
can see no clear recent changes in total area after an initial increase
in the 1990s which is likely a reflection of changes in fleet dynamics.

\textbf{I think that monitoring the distribution of catch of species
could be useful in this report, however I feel total area may be an
insensitive indicator in this regard. Maybe we can ID a more
appropriate/sensitive indicator? Maybe this is covered by the COG or
inertia indicator? I have explored other potential options here in the
proportion of PS catch \textgreater{} 180 degrees and proportion of LL
catch outside 20 degrees}

\begin{figure}
\centering
\includegraphics{./2024_analyses/results/catch/catch_area_PS.png}
\caption{Total area of purse seine catch by species and year. Black line
= 1990:2000 mean.}
\end{figure}

\clearpage

\begin{figure}
\centering
\includegraphics{./2024_analyses/results/catch/catch_area_LL.png}
\caption{Total area of longline catch by species and year.}
\end{figure}

\clearpage

\hypertarget{proportion-of-ps-catch-180-degrees-longitude}{%
\subsection{Proportion of PS catch \textgreater{} 180 degrees
longitude}\label{proportion-of-ps-catch-180-degrees-longitude}}

This is another indicator looking at monitoring the predicted shifts in
tuna biomass eastwards with climate change and expansion of the warm
pool. In previous SPC reports, this indicator has looked at the
proportion of catch/sets in the high seas which is a similar indicator
and relevant to members as they derive income from fishing within their
EEZs. However, I have switched to a longitudinal cutoff instead as it is
simpler and should be more reflective of a climate change shift as the
high seas occur in various regions (not all eastern) in the WCPFC. Not
sure if 180 degrees is the best boundary to use?

\begin{figure}
\centering
\includegraphics{./2024_analyses/results/catch/catch_180prop_PS.png}
\caption{Proportion of catch of tuna species beyond 180 degrees
longitude. Data is normalised by dividing each index by the historical
mean (1990:2000).}
\end{figure}

\clearpage

\hypertarget{proportion-of-ll-catch-beyond-20-degrees-latitude}{%
\subsection{Proportion of LL catch beyond 20 degrees
latitude}\label{proportion-of-ll-catch-beyond-20-degrees-latitude}}

The intent of this indicator is to identify poleward shifts in species
and the poleward shift of effort and catch in response. There are many
documented cases of species shifting polewards with warming waters as a
response to climate change, such as various pelagic species along
Australia's east coast as an example. This is more relevant for longline
fisheries than the purse seine fleet, and could track poleward shifts in
more temperate species like albacore. Could also look to explore an
indicator species like a mako shark? porbeagle? striped marlin? etc.

However, I am not sure that a) this indicator is sensitive, and b) that
the fleet will follow species this way and that other dynamics may be
more influential. We can see in the plot it seems like their is a
`tropicalisaiton' of catch, which is likely reflecting a change in fleet
dynamics or targeting rather than reflecting species distributions.

\begin{figure}
\centering
\includegraphics{./2024_analyses/results/catch/catch_20prop_LL.png}
\caption{Proportion of longline catch by species harvested outside -20
to 20 degrees latitude by year.}
\end{figure}

\clearpage

\hypertarget{effort-indicators}{%
\section{Effort indicators}\label{effort-indicators}}

Here, we have extracted aggregated effort data for purse seine (by set
type) and longline fisheries in the WCPFC from 1990-2023. We investigate
a series of potential indicators in effort that are similar to those for
catch and how we might monitor the fishery using this data. This
includes:

\begin{itemize}
\item
  Centre of gravity of PS effort,
\item
  Total area of PS and longline effort,
\item
  Proportion of PS effort \textgreater{} 180 degrees longitude.
\end{itemize}

\hypertarget{centre-of-gravity-cog-of-purse-seine-effort}{%
\subsection{Centre of gravity (COG) of purse seine
effort}\label{centre-of-gravity-cog-of-purse-seine-effort}}

Centre of gravity refers to the `centre' of fishing effort in terms of
both longitude and latitude. With the effects of climate change, is is
predicted that tuna biomass will shift eastwards with the expansion of
the warm pool. By monitoring the centre of gravity of effort, we can
track if this shift in biomass is occurring (assuming the fleet will
follow the biomass). It is also well known that effort and catch shifts
in relation to ENSO events, so it could shift with other
environmental-related shifts like climate change.

Here, we extract the centre of gravity of effort which basically
provides a ``mean'' estimate of effort (longitude and latitude) and an
intertia value which measures disperson (basically spread). We have
extracted these at annual timescales, but could be downgraded to
seasonal or monthly to explore within-year shifts also. Note: data is
filtered to WCPFC regions 6-8 (lat: -20:10 degrees, lon 140:210 degrees)
to capture main PS fleet and filter out SE Asian domestic effort (right
approach? to be discussed).

First, we show a plot of effort density as a way of a summary of effort
spatially comparing purse seine effort from a historical period
(1990:2000) and recent years. Then, we look at COG by set type and can
see that the PS fleet shifts east and west which is known to occur with
ENSO events, and that there is a clear eastward shift in associated sets
while unassociated sets seem more variable.

\begin{figure}
\centering
\includegraphics{./2024_analyses/results/effort/effort_density_PSall_region68.png}
\caption{Density map of purse seine effort in historical and recent
years.}
\end{figure}

\clearpage

\begin{figure}
\centering
\includegraphics{./2024_analyses/results/effort/effort_COGPS_map_region68_set_type.png}
\caption{Centre of gravity of purse seine effort by set type from
1990-2023.}
\end{figure}

\clearpage

\hypertarget{total-area-of-effort}{%
\subsection{Total area of effort}\label{total-area-of-effort}}

This indicator monitors the total area of purse seine and longline
effort by year. Area that a fleet fishes often changes over time in
response to changing fishing dynamics, stock abundance, fishing pressure
etc. It is not certain how climate change will influence species in this
regard, but this is one way we could look to monitor any shifts. Similar
to the total area catch indicator, we can see an increase in early years
followed by relative stability in total area for PS effort since
\textasciitilde2000. As discussed above in the catch indicator, I think
a `footprint' type indicator could be useful to monitor however open to
suggestions of more sensitive indicators. Is inertia a good alternative?
For longline, we can see an even more stable (uninformative) trend over
time. It should be noted that longline effort has a low resolution (5x5
degrees) and so this indicator is likely to be truly informative.
Alternative? I have added a plot of PS effort inertia for discussion.

\begin{figure}
\centering
\includegraphics{./2024_analyses/results/effort/effort_area_PS.png}
\caption{Total area of purse seine effort by set type from 1990-2023.}
\end{figure}

\begin{figure}
\centering
\includegraphics{./2024_analyses/results/effort/effort_area_LL.png}
\caption{Total area of longline effort by set type from 1990-2023.}
\end{figure}

\begin{figure}
\centering
\includegraphics{./2024_analyses/results/effort/effort_COGinert_PSs_region68.png}
\caption{Inertia (spread) of purse seine effort by set type from
1990-2023.}
\end{figure}

\clearpage

\hypertarget{proportion-of-purse-seine-sets-180-degrees}{%
\subsection{Proportion of purse seine sets \textgreater{} 180
degrees}\label{proportion-of-purse-seine-sets-180-degrees}}

As above with catch, we look at the proportion of PS sets occurring east
of the 180 degree longitude. It has been predicted that an eastward
shift in tuna biomass will occur with climate change and this indicator
could identify this shift. There are a range of factors that drive fleet
dynamics and this indicator could help track if this shift in
biomass/distribution is occurring, and if the fleet is shifting with it.
This indicator also seems to fluctuate potentially in line with ENSO
events. As above, is 180 degrees the best longitude? Is this preferable
to proportion of shots in the high seas?

\begin{figure}
\centering
\includegraphics{./2024_analyses/results/effort/effort_180prop_PS_region68.png}
\caption{Change in the proportion of sets beyond 180 degrees longitude
by set type relative to their historical (1990:2000) mean from
1990-2023.}
\end{figure}

\clearpage

\hypertarget{biological-indicators}{%
\section{Biological indicators}\label{biological-indicators}}

Climate change is expected to modify species life history and biology in
a range of ways from maximum size to size at maturity etc. by monitoring
biological indicators, we may be able to identify changes in tuna
population related to climate change. Here, we have extracted biological
information from a range of SPC data sources (eg: observer, port
sampling etc) to be used to inform various biological (length,
length/weight) indicators. Data has been sourced from longline gear only
as this provides a broader spread of biological information (small-large
fish) compared to PS. The primary biological data collected by fisheries
dependent sampling by SPC and WCPFC is length-based while other
information like weight, age, maturity etc is less frequently collected.
For this reason, the focus of these indicators are largely length-based.

\begin{itemize}
\item
  Mean relative condition factor of tuna species (BET, SKJ, YFT),
\item
  Mean length of tuna species (BET, SKJ, YFT),
\item
  Proportional length indicators of tuna species (BET, SKJ, YFT).
\end{itemize}

Species growth and sizes are influenced by a range of factors such as
prey availability and water temperature, both of which are predicted to
change with climate change. These indicators would attempt to monitor
changes in the biology and structure of populations to see if they're
changing over time. It will be important to consider that changes in
targeting and fleet dynamics, fishing pressure, sampling design and
robustness, and the data available will also all affect trends in these
indicators. Many biological indicators, for example mean length are
quite insensitive and require large shifts in the abundance/structure of
a stock to record a change and are more likely to show shifts in fishing
or sampling rather than the stock. Other indicators that could be worth
considering monitoring or collecting data for in the future could be
maturity/fecundity which also may show shifts with climate change and
may be more sensitive.

\hypertarget{mean-relative-condition-factor-of-tuna-species-bet-skj-yft}{%
\subsection{Mean relative condition factor of tuna species (BET, SKJ,
YFT)}\label{mean-relative-condition-factor-of-tuna-species-bet-skj-yft}}

The intent of this indicator is to identify changes in the condition
(ie: weight relative to length) of tuna species over time. Species
growth and sizes are influenced by a range of factors such as prey
availability and water temperature, both of which are predicted to
change with climate change. This indicator looks at the length-weight
relationship of species and sees if it is shifting over time. ie: are
fish getting skinnier/fatter over time?

This indicator comes from existing SPC reports and is a clever use of
existing data and an interesting way of exploring a change in a
population. However, I am not sure if we have applied it in the best way
possible to identify shifts. At the moment, it fits a length-weight
curve to data for all years and then divides the observed weight by the
predicted to get a ratio value looking at the deviation of observed vs
predicted. I think this is a somewhat insensitive way to monitor this
and could explore alternate designs. Any ideas? Would it not be better
to fit a curve to the early data, and then do a similar exercise? If all
data is used to inform the curve it will always center near 1\ldots{} I
have explored this briefly by fitting a length-weight curve by year, and
to historical (1990:2000) vs recent years. I also explored extracting
observed vs predicted weights for fish of various lengths and plotting
these to see if a 100cm fish has changed in weight over time as an
example (need to look into this, used predicted weights, not observed).

\begin{figure}
\centering
\includegraphics{./2024_analyses/results/biology/biology_lenwt_indicator.png}
\caption{Relative condition factor of tunas from 1990:2023 identifying
changes in the lenght-weight relationship over time.}
\end{figure}

\begin{figure}
\centering
\includegraphics{./2024_analyses/results/biology/biology_lenwt_hist_recent_indicator.png}
\caption{Length-weight relationship of tuna species over time with the
historical period (1990-2000) shaded in grey, and recent years coloured
lines.}
\end{figure}

\begin{figure}
\centering
\includegraphics{./2024_analyses/results/biology/biology_lenwt_species_lenquants2.png}
\caption{Exploring changes in the weight of different lengthed fish over
time for different tunas.}
\end{figure}

\clearpage

\hypertarget{mean-length-of-tuna-species}{%
\subsection{Mean length of tuna
species}\label{mean-length-of-tuna-species}}

Here we look at length data from longline gear from a range of sampling
programs (port, sampling etc). Changes in a species length composition
could reflect changes in stock composition such as a recruitment pulse,
or broader changes in abundance. Mean length is a simple indicator to
monitor for stakeholders and there is alot of length data collected in
the WCPFC so is accessible. But, it is likely to be a relatively
insensitive indicator to changes in a stock and is more likely to
reflect changes in targeting or sampling for example. Note, there is a
slight downward trend in YFT mean length.

\begin{figure}
\centering
\includegraphics{./2024_analyses/results/biology/biology_mnlen_indicator2.png}
\caption{Mean length of tunas from 1990:2023 from longline gear.}
\end{figure}

\clearpage

\hypertarget{proportional-length-indicators-of-tuna-species}{%
\subsection{Proportional length indicators of tuna
species}\label{proportional-length-indicators-of-tuna-species}}

This indicator looks at the proportion of different-sized fish in the
catch over time. Changes in the size composition of a population can
reflect change. For example, when fishing pressure is high often the
proportion of large individuals in a population can decrease. Monitoring
small individuals can also help determine if recruitment is changing
which could be influenced by climate change. This is potentially a more
sensitive indicator than mean length and better captures if the size
structure of the population and catch is changing. In this plot we can
see that the proportion of large yellowfin tuna (\textgreater137cm) in
the longline catch is decreasing, and the proportion of small
individuals (\textless105cm) increasing. Is this a change in fishing
behaviour or a change in the stock?

\begin{figure}
\centering
\includegraphics{./2024_analyses/results/biology/biology_lenprop_indicator2.png}
\caption{Proportion of different-sized fish in the catch from 1990-2023
for tuna species. Red = Proportion of catch above the mean size, blue =
proportion of catch below the small size, green = proportoin of catch
above the large size.}
\end{figure}

\clearpage

\hypertarget{length-composition-exploration}{%
\subsection{Length composition
exploration}\label{length-composition-exploration}}

Lastly, here are a couple of plots inspired by the SAFE reports from the
WPRFMC that I think capture well the length composition of the tunas
over time.

\begin{figure}
\centering
\includegraphics{./2024_analyses/results/biology/biology_lenridges_indicator1.png}
\caption{Size composition of tunas from 1990-2023 from longline.}
\end{figure}

\begin{figure}
\centering
\includegraphics{./2024_analyses/results/biology/biology_lenridges_indicator3.png}
\caption{Size composition of tunas in historical years (1990:2000) and
the most recent year (2023) for longline.}
\end{figure}

\end{document}
